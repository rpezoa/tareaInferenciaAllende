\documentclass[10pt, conference, compsocconf]{IEEEtranBST/IEEEtran}
\usepackage[cmex10]{amsmath}
\usepackage{amsfonts}
\usepackage{txfonts}
\usepackage{graphicx}
%\usepackage[latin 1]{input enc}
\usepackage{graphicx}
\usepackage{subfigure}
\usepackage{pgf}
\usepackage{tabularx}
\usepackage{supertabular}
\usepackage{color}
%\usepackage[centertags]{amsmath}
%\usepackage{amsfonts}
%\usepackage{amssymb}
%\usepackage{mathrsfs}
%\usepackage{pstricks}
%\usepackage{fancyvrb}
%\usepackage[spanish, english]{babel}
%\usepackage[utf8]{inputenc}
%\usepackage{tabularx}
%\usepackage{supertabular}
%\usepackage{gensymb}
%\usepackage{esvect}
%\usepackage{pgf}



\definecolor{red}{rgb}{1,0,0}
\newcommand{\red}{\textcolor{red}}

\hyphenation{op-tical net-works semi-conduc-tor}

\pgfdeclareimage[width=7cm]{ihc1}{img/resultado1}

\begin{document}
%
% paper title
% can use linebreaks \\ within to get better formatting as desired
\title{Tarea N° 1}


% author names and affiliations
% use a multiple column layout for up to two different
% affiliations

\author{\IEEEauthorblockN{Raquel Pezoa}
\IEEEauthorblockA{\\Departamento de Inform\'{a}tica\\ Universidad T\'ecnica Federico Santa Mar\'ia, (UTFSM)\\
Valpara\'iso, Chile\\
\\ Email: raquel.pezoa@usm.cl }
}


% make the title area
\maketitle


\IEEEpeerreviewmaketitle


\begin{enumerate}
\item \textbf{t-norma}
Verificar que los operadores PAND (t-norm probabilistic) y LAND (t-norm de Lukasiewics) cumplen las propiedades t-norma.\\

Let $X$ denotes the universal set. A fuzzy set $A$ is defined by a membership function $\mu_{A}:x \rightarrow [0,1]$ that describes the membership degree of the elements in X. Values of $\mu_{A}$ closer to 1 denote higher degree of membership. Therefore, the degree in which the statement \emph{$x$ is in $A$} is true is determined by the ordered pair $(x,\mu_{A}(x)$.

Fundamental operations used in classical set theory, like intersections and unions, are modeled in fuzzy logic by triangular norms and triangular conorms, respectively. Intersections are modeled using triangular norms (t-norms). A t-norm based intersection corresponds to the mapping $T:[0,1] \times [0,1] \rightarrow [0,1]$ and has been extensively used to model the logical connective \emph{and}.
Meanwhile, triangular conorms (s-norms), given by the mapping $S:[0,1] \times [0,1] \rightarrow [0,1]$, are used to model the logical connective \emph{or}.

Any t-norm must satisfy the properties:
\begin{itemize}
\item Simmetry: $T(x,y)=T(y,x)$
\item Associativity: $T(x,T(y,z))=T(T(x,y),z)$
\item Monotonicity: $T(x,y) \leq T(x',y')$ if $x\leq x'$ and $y\leq y'$
\item One identitiy: $T(x,1)=x$
\end{itemize}
$\forall \ x,x',y,y',z \in [0,1]$

\begin{enumerate}
\item $PAND(x,y)=xy$
\begin{itemize}
\item Simmetry:
$PAND(x,y)=PAND(y,x)?$

$PAND(x,y)=xy$ , con $x,y \in [0,1]$ y por propiedad conmutativa de la multipliacación $xy=yx=PAND(y,x)$, por lo tanto, $PAND(x,y)=PAND(y,x)$

\item Associativity 
$PAND(x,PAND(y,z))=PAND(PAND(x,y),z)$

$PAND(x,PAND(y,z))=PAND(x,yz)=x(yz)$,  con $x,y,y \in [0,1]$ y por propiedad asociativa de la multiplicación se tiene que $x(yz) = (xy)z$, por lo tanto $x(yz)=(xy)z=PAND(PAND(x,y),z)$

\item Monotonocity
$PAND(x,y) \leq PAND(x',y')$ if $x\leq x'$ and $y\leq y'$

$xy \leq x'y'$

\end{itemize}
\item $LAND(x,y)=max(x+y-1,0)$
\end{enumerate}


\item s-norma
Verificar que los operadores POR (s-norm probabilistic) y LOR (s-norm de Lukasiewics) cumplen las propiedades s-norma.


Any s-norm must satisfy the properties:
\begin{itemize}
\item Simmetry: $S(x,y)=S(y,x)$
\item Associativity: $S(x,S(y,z))=S(S(x,y),z)$
\item Monotonicity: $S(x,y) \leq S(x',y')$ if $x\leq x'$ and $y\leq y'$
\item Zero identitiy: $S(x,0)=x$
\end{itemize}
$\forall \ x,x',y,y',z \in [0,1]$

$$POR(x,y)=x+y-xy$$
$$LOR(x,y)=min(x+y,1)$$

\item Propiedades Fuzzy Sets
Verificar si las siguientes propiedades se cumplen para Fuzzy sets. Use las t-norm y s-norm anteriores; en caso de no ser v\'{a}lida la propiedad mostrar con un contraejemplo.
$$(A B)= A B$$
$$A (B C)= (A B)(A C)$$
$$A A = X$$
$$A A= $$

\item Relaciones Fuzzy
Sean $R$ y $S$ relaciones fuzzy de $X$ en $Y$. Verificar que las relaciones Fuzzy cumplen las siguientes propiedades. Use las t-norm y s-norm de Mandani.

\item Sean $A$, $A'$ Fuzzy sets y sean $R$, $R'$ relaciones Fuzzy de $X$ en $Y$ y $S$, $T$ relaciones Fuzzy de $Y$ en $Z$. Verificar que:

\end{enumerate}
\end{document}


